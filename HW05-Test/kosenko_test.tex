\documentclass[a4paper,12pt]{article}
\usepackage{cmap}					% поиск в PDF
\usepackage{mathtext} 				% русские буквы в формулах
\usepackage[T2A]{fontenc}			% кодировка
\usepackage[utf8]{inputenc}			% кодировка исходного текста
\usepackage[english,russian]{babel}	% локализация и переносы
\usepackage{indentfirst}
\frenchspacing
\usepackage{amsmath,amsfonts,amssymb,amsthm,mathtools} % AMS
\usepackage{icomma} % "Умная" запятая: $0,2$ --- число, $0, 2$ --- перечисление
\DeclareMathOperator{\sgn}{\mathop{sgn}}
\newcommand*{\hm}[1]{#1\nobreak\discretionary{}
{\hbox{$\mathsurround=0pt #1$}}{}}
\usepackage{graphicx}  % Для вставки рисунков
\graphicspath{{images/}{images2/}}  % папки с картинками
\setlength\fboxsep{3pt} % Отступ рамки \fbox{} от рисунка
\setlength\fboxrule{1pt} % Толщина линий рамки \fbox{}
\usepackage{wrapfig} % Обтекание рисунков текстом
\usepackage{array,tabularx,tabulary,booktabs} % Дополнительная работа с таблицами
\usepackage{longtable}  % Длинные таблицы
\usepackage{multirow} % Слияние строк в таблице
\theoremstyle{plain} % Это стиль по умолчанию, его можно не переопределять.
\newtheorem{theorem}{Теорема}[section]
\newtheorem{proposition}[theorem]{Утверждение}
\theoremstyle{definition} % "Определение"
\newtheorem{corollary}{Следствие}[theorem]
\newtheorem{problem}{Задача}[section]
\theoremstyle{remark} % "Примечание"
\newtheorem*{nonum}{Решение}
\usepackage{etoolbox} % логические операторы
\usepackage{extsizes} % Возможность сделать 14-й шрифт
\usepackage{geometry} % Простой способ задавать поля
	\geometry{top=25mm}
	\geometry{bottom=35mm}
	\geometry{left=35mm}
	\geometry{right=20mm}
\usepackage{setspace} % Интерлиньяж
\usepackage{lastpage} % Узнать, сколько всего страниц в документе.
\usepackage{soul} % Модификаторы начертания
\usepackage{hyperref}
\usepackage[usenames,dvipsnames,svgnames,table,rgb]{xcolor}
\hypersetup{unicode=true, pdftitle={Заголовок}, pdfauthor={Автор},pdfsubject={Тема}, pdfcreator={Создатель}, pdfproducer={Производитель}, pdfkeywords={keyword1} {key2} {key3}, colorlinks=true, linkcolor=red, citecolor=black, filecolor=magenta, urlcolor=cyan}
\usepackage{csquotes} % Еще инструменты для ссылок
\usepackage{multicol} % Несколько колонок
\usepackage{tikz} % Работа с графикой
\usepackage{pgfplots}
\usepackage{pgfplotstable}

\date{ВШЭ ПМИ, 6 вариант, 2022 г.} 
\author{Алексей Косенко}


\begin{document}

\maketitle

\section{Язык Дика.}

Язык, в котором открывающая скобка не может встречаться сразу после закрывающей скобки такого же вида. Запишем грамматику с одним нетерминалом.

$$G = \left< \{(, ), [, ]\}, \{S\}, P, S \right>$$
$$S \rightarrow ( \ S \ ) \ \ | \ \ [ \ S \ ] \ \  | \ \ ( \ S \ ) \ [ \ S \ ] \ \ | \ \ [ \ S \ ] \ ( \ S \ ) \ \ | \ \ \varepsilon$$

Грамматика является языком Дика с двумя видами скобок, и она верно задает язык. Вложенность может быть любой, то есть мы можем построить ПСП, заключенную в два вида скобок. Также справедлива конкатенация двух ПСП, которые заключены в различные виды скобок. Конкатенировать ПСП, заключенные в скобки одного и такого вида нам запрещено по условию.

Можем записать эквивалентную грамматику, задающую тот же самый язык, с тремя нетерминалами.

$$S \rightarrow ( \ S \ ) \ A \ \ | \ \ [ \ S \ ] \ B \ \ | \ \ \varepsilon $$
$$A \rightarrow [ \ S \ ] \ B \ \  | \ \ \varepsilon $$
$$B \rightarrow ( \ S \ ) \ A \ \ | \ \ \varepsilon $$

\section{Вывод полученной грамматики.}
Приведем левосторонний вывод скобок двух видов и пустой цепочки.

$$S \rightarrow ( \ S \ ) \rightarrow^{\varepsilon} ( \ )$$
$$S \rightarrow [ \ S \ ] \rightarrow^{\varepsilon} [ \ ]$$
$$S \rightarrow^{\varepsilon} \varepsilon$$

Любые цепочки, которые не являются ПСП, не выводятся из нашей грамматики. Также мы не можем получить конкатенацию двух ПСП, которые заключены в скобки одного и такого же вида.

$$( \ [ \ ( \ ) \ ] \ ( \ ) \ ) \ ( \ )$$
$$( \ ] \ [ \ )$$
$$[ \ ] \ [ \ ]$$

\section{LL-анализ.}
Запишем множество FIRST и FOLLOW.

$$FIRST = \{ (, [, \varepsilon \}$$
$$FOLLOW = \{ ), ], \$ \}$$

\end{document}
